\section{Limits at Infinity; Horizontal Asymptotes}
\setcounter{theorem}{7}
\begin{definition}
	Let $f$ be a function defined on some interval $\left( -\infty, a \right)$. Then
	\begin{equation*}
	\lim\limits_{x \to -\infty} f\left(x\right) = L
	\end{equation*}
	means that for every $\epsilon > 0$ there is a corresponding number $N$ such that if $x < N$ then $\left| f\left(x\right) - L \right| < \epsilon$.
\end{definition}

\subsection{Exercises}
\setcounter{exercise}{70}
\begin{exercise}
	Use Definition~8 to prove that $\lim\limits_{x \to \infty} \frac{1}{x} = 0$.
\end{exercise}

