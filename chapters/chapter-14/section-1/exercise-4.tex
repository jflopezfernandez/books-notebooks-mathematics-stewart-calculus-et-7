\begin{exercise}
	\index{function!Cobb-Douglas}
	Verify for the Cobb-Douglas production function
	\begin{equation*}
	P\left( L, K \right) = 1.01L^{0.75}K^{0.25}
	\end{equation*}
	discussed in Example 3 that the production will be doubled if both the amount of labor and the amount of capital are doubled. Determine whether this is also true for the general production function:
	\begin{equation}
	\label{equation-Cobb-Douglas}
	P\left( L, K \right) = bL^{\alpha}K^{1-\alpha}
	\end{equation}
\end{exercise}
\begin{solution}
	We will analyze the general Cobb-Douglas production function to determine whether a doubling of both labor and capital results in a doubling of production.
	\begin{align*}
	2 P\left( L, K \right) &\overset{?}{=} b \left( 2L \right)^{\alpha} \left( 2K \right)^{1 - \alpha} \\
	&\overset{?}{=} b2^{\alpha}L^{\alpha}2^{1-\alpha}K^{1-\alpha} \\
	&\overset{?}{=} \left( 2^{\alpha} \cdot 2^{1-\alpha} \right) bL^{\alpha}K^{1-\alpha} \\
	&\overset{?}{=} \left( 2^{\left( \alpha \right) + \left( 1 - \alpha \right)} \right) bL^{\alpha}K^{1-\alpha} \\
	&\overset{?}{=} \left( 2^{1} \right) bL^{\alpha}K^{1-\alpha} \\
	&\overset{?}{=} 2bL^{\alpha}K^{1-\alpha}
	\end{align*}
	Indeed, we can conclude that for any arbitrary Cobb-Douglas production function $P : \mathbb{R}^{+} \times \mathbb{R}^{+} \to \mathbb{R}$, a doubling of both the labor and capital input parameters will result in a doubling of the roduction output.
\end{solution}
