\begin{exercise}
	In Example 2 we considered the function $W = f\left( T, v \right)$, where $W$ is the wind-chill index, $T$ is the actual temperature, and $v$ is the wind speed. A numerical representation is given in Table 1.
	\renewcommand{\labelenumi}{(\alph{enumi})}
	\begin{enumerate}
		\item What is the value of $f\left(-15,40\right)$? What is its meaning?
		\item Describe in words the meaning of the question ``For what value of $v$ is $f\left(-20,v\right)=-30$?'' Then answer the question.
		\item Describe in words the meaning of the question ``For what value of $T$ is $f\left(T,20\right)=-49$?'' Then answer the question.
		\item What is the meaning of the function $W = f\left(-5,v\right)$? Describe the behaviour of this function.
		\item What is the meaning of the function $W = f\left(T,50\right)$? Describe the behaviour of this function.
	\end{enumerate}
\end{exercise}
\begin{solution}
	{\color{white}{content...}}
	\renewcommand{\labelenumi}{(\alph{enumi})}
	\begin{enumerate}
		\item $f\left(-15,40\right) = -27$. This means that when the temperature is $-15$ degrees Celsius and the wind speed is $40$ km/hr, the temperature outside feels more like $-27$ degrees Celsius, according to Table 1.
		\item The question is asking, ``How hard would the wind have to be blowing (measured in terms of km/hr) to make the actual temperature of $-20$ degrees Celsius feel like $-30$ degrees Celsius?'' The answer is $20$ km/hr.
		\item The question is asking, ``How cold would it have to be outside for $20$ km/hr winds to make it feel like $-49$ degrees Celsius outside?'' The answer is $-35$ degrees Celsius.
		\item This function describes the wind chill $W$ with respect to wind speed $v$, as the actual temperature is held constant at $-5$ degrees Celsius. The function is therefore reduced to a simple $1$-variable negative linear function.
		\item This function describes the wind chill $W$ in terms of the actual temperature $T$, as wind speed is held constant at $50$ km/hr. The function is thus a simplified $1$-variable negative linear function.
	\end{enumerate}
\end{solution}
