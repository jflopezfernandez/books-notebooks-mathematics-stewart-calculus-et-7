\begin{exercise}
	The \textit{temperature-humidity index} $I$ (or humidex, for short) is the perceived air temperature when the actual temperature is $T$ and the relative humidity is $h$, so we can write $I = \left( T, h \right)$.
	\renewcommand{\labelenumi}{(\alph{enumi})}
	\begin{enumerate}
		\item What is the value of $f\left( 95, 70 \right)$? What is its meaning?
		\item For what value of $h$ is $f\left( 90, h \right) = 100$?
		\item For waht value of $T$ is $f\left( T, 50 \right) = 88$?
		\item What are the meanings of the functions $I = f\left( 80, h \right)$ and $I = f\left( 100, h \right)$? Compare the behavior of these two functions of $h$.
	\end{enumerate}
\end{exercise}
\begin{solution}
	{\color{white}{content...}}
	\renewcommand{\labelenumi}{(\alph{enumi})}
	\begin{enumerate}
		\item The value of $f\left( 95, 70 \right)$ is $124$, meaning that when the actual temperature is $95^{\circ}$F and the relative humidity is at $70\%$, the temperature will feel like $124^{\circ}$F.
		\item When $h = 60\%$, $f\left( 90, h \right) = 100$.
		\item When $T = 85^{\circ}$, $f\left( T, 50 \right) = 88$.
		\item Both $I_1 = f\left( 80, h \right)$ and $I_2 = f\left( 100, h \right)$ are $1$-variable functions relating the relative humidity to the temperature-humidity index. Since the temperature is constant and different for $I_1$ and $I_2$, $I_1\left( h \right) \neq I_2\left( h \right)$ for any value of $h$. Both $I_1$ and $I_2$ are linear functions of $h$.
	\end{enumerate}
\end{solution}
