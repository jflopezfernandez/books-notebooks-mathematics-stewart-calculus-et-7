\section{Line Integrals}
\begin{definition}
	\index{line integral}
	\label{definition-16.2.2}
	If $f$ is defined on a smooth curve $C$ given by Equation~1, then the \textbf{line integral of $f$ along $C$} is
	\begin{equation}
	\int_C f\left(x,y\right)ds = \lim\limits_{n \to \infty} \sum_{i=1}^{n} f\left(x_i^*,y_i^*\right)\Delta s_i
	\end{equation}
	if this limit exists.
\end{definition}
\begin{definition}
	If $f$ is a continuous function, then the limit in Definition~\ref{definition-16.2.2} always exists and the following formula can be used to evaluate the line integral.
	\begin{equation}
	\int_C f\left(x,y\right)ds = \int_a^b f\left(x(t),y(t)\right)\sqrt{\left(\frac{dx}{dt}\right)^2 + \left(\frac{dy}{dt}\right)^2} \ dt
	\end{equation}
\end{definition}
\begin{definition}
	Let $\mathbf{F}$ be a continuous vector field defined on a smooth curve $C$ given by a vector function $\mathbf{r}\left(t\right)$, $a \leq t \leq b$. Then the \textbf{line integral of $\mathbf{F}$ along $C$} is
	\begin{equation}
	\int_C \mathbf{F} \cdot d\mathbf{r} = \int_a^b \mathbf{F}\left(\mathbf{r}\left(t\right)\right) \  \mathbf{\cdot} \  \mathbf{r}^{\prime}\left(t\right) dt = \int_C \mathbf{F} \ \mathbf{\cdot} \ \mathbf{T} \  ds
	\end{equation}
\end{definition}

\subsection{Exercises}
\begin{exercise}
	$\displaystyle\int_C y^3 ds, \ C:x=t^3, \ y=t, \ 0\leq t \leq 2$
\end{exercise}
\begin{solution}
	The equation for the line integral of $f\left(x,y\right)$ over $C$ is
	\begin{equation*}
	\int_C f\left(x,y\right) ds = \int_a^b f\left(x(t),y(t)\right) \sqrt{\left(\frac{dx}{dt}\right)^2 + \left(\frac{dy}{dt}\right)^2}\ dt.
	\end{equation*}
	Therefore, we begin by converting $f\left(x,y\right)$ into a function of $t$.
	\begin{align*}
	f\left(x,y\right) &= f\left(x(t),y(t)\right) \\
	&= t^3
	\end{align*}
	We now proceed by finding the partial derivatives of $x(t)$ and $y(t)$ with respect to $t$.
	\begin{align*}
	\frac{dx}{dt} &= \frac{d}{dt}\left(x(t)\right) \\
	&= \frac{d}{dt}\left(t^3\right) \\
	&= 3t^2
	\end{align*}
	\begin{align*}
	\frac{dy}{dt} &= \frac{d}{dt}\left(y(t)\right) \\
	&= \frac{d}{dt}\left(t\right) \\
	&= 1
	\end{align*}
	With the above three components in hand, we simply plug in these values into the line integral formula above and integrate.
	\begin{align*}
	\int_C f\left(x,y\right) ds &= \int_a^b f\left(x(t),y(t)\right)\sqrt{\left(\frac{dx}{dt}\right)^2+\left(\frac{dy}{dt}\right)^2}\ dt \\
	&= \int_0^2 t^3 \sqrt{\left(3t^2\right)^2 + \left(1\right)^2} \ dt \\
	&= \int_0^2 t^3 \sqrt{ 9t^4 + 1 }\ dt \\
	\intertext{Let $u = 9t^4 + 1$. Then $du = 36t^3dx$ and $\frac{1}{36}du = t^3dx$. The limits of integration become $u(0) = 9(0)^4+1 = 1$ and $u(2)=9(2)^4+1=145$.} \\
	&= \frac{1}{36} \int_{1}^{145} \sqrt{u} \ du \\
	&= \frac{1}{36} \cdot \left[ \frac{2}{3}u^{\frac{3}{2}} \right]_1^{145} \\
	&= \frac{1}{36} \cdot \left[ \left( \frac{2}{3}\left(145\right)^{\frac{3}{2}} \right) - \left( \frac{2}{3} \left(1\right)^{\frac{3}{2}} \right) \right] \\
	&= \frac{1}{54} \left( 145^{\frac{3}{2}} - 1 \right)
	\end{align*}
\end{solution}

\begin{exercise}
	Evaluate the line integral, where $C$ is the given curve.
	\begin{equation*}
	\int_C xy \ ds, \quad C: x=t^2, \ y=2t, \ 0 \leq t \leq 1
	\end{equation*}
\end{exercise}
\begin{solution}
	The equation for the line integral of a curve $C$ is
	\begin{equation*}
	\int_C f\left(x,y\right) \ ds = \int_a^b f\left(x(t),y(t)\right)\sqrt{\left(\frac{dx}{dt}\right)^2+\left(\frac{dy}{dt}\right)^2}\ dt.
	\end{equation*}
	We thus begin by calculating the first partial derivatives with respect to $t$.
	\begin{align*}
	\frac{dx}{dt} &= \frac{d}{dt}\left(x(t)\right) \\
	&= \frac{d}{dt}\left(t^2\right)\\
	&= 2t
	\end{align*}
	\begin{align*}
	\frac{dy}{dt} &= \frac{d}{dt}\left(y(t)\right) \\
	&= \frac{d}{dt}\left(2t\right)\\
	&= 2
	\end{align*}
	We now plug in these values into the line integral equation, integrate, and simplify.
	\begin{align*}
	\int_C f\left(x,y\right) &= \int_a^b f\left(x(t),y(t)\right) \sqrt{ \left(\frac{dx}{dt}\right)^2 + \left(\frac{dy}{dt}\right)^2}\ dt \\
	&= \int_0^1 \left(t^2\right)\left(2t\right)\sqrt{\left(2t\right)^2+\left(2\right)^2}\ dt \\
	&= 2 \int_0^1 t^3 \sqrt{ 4t^2 + 4 }\ dt \\
	&= 4 \int_0^1 t^3 \sqrt{ t^2 + 1 }\ dt \\
	\intertext{Let $u = t^2 + 1$. Then $du = 2t\ dt$ and $t^2 = u - 1$.} \\
	&= 2 \int_1^2 \left( u - 1 \right)\left(\sqrt{u}\right)\ du \\
	&= 2 \left[ \int_1^2 u^{\frac{3}{2}}\ du - \int_1^2 u^{\frac{1}{2}}\ du \right] \\
	&= 2 \cdot \left( \left[ \frac{2}{5} u^{\frac{5}{2}} \right]_1^2 - \left[ \frac{2}{3} u^{\frac{3}{2}} \right]_1^2 \right) \\
	&= 2 \left( \left( \frac{2}{5} \left(2\right)^{\frac{5}{2}} - \frac{2}{5} \left(1\right)^{\frac{5}{2}} \right) - \left( \frac{2}{3} \left(2\right)^{\frac{3}{2}} - \frac{2}{3} \left(1\right)^{\frac{3}{2}} \right) \right) \\
	&= 4 \left[ \frac{1}{5} \left( 4\sqrt{2} - 1 \right) - \frac{1}{3} \left( 2\sqrt{2} - 1 \right) \right] \\
	&= \frac{4}{5}\left(4\sqrt{2} - 1\right) - \frac{4}{3}\left(2\sqrt{2} - 1\right) \\
	&= \frac{16\sqrt{2}}{5} - \frac{4}{5} - \frac{8\sqrt{2}}{3} + \frac{4}{3} \\
	&= \frac{48\sqrt{2}}{15} - \frac{12}{15} - \frac{40\sqrt{2}}{15} + \frac{20}{15} \\
	&= \frac{8+8\sqrt{2}}{15}
	\end{align*}
\end{solution}

