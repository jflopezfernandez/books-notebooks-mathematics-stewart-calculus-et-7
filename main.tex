
\documentclass{book}

\usepackage[utf8]{inputenc}
\usepackage[T1]{fontenc}
\usepackage[english]{babel}
\usepackage[
paperheight=8.5in,
paperwidth=5.5in,
outer=0.5in,
inner=0.6in,
bottom=1in,
top=0.7in
]{geometry}

\setlength{\parindent}{0em}

\usepackage{xcolor}
\newcommand{\todo}[1]{{\color{red}{TODO: #1}}}
\newcommand{\proofincomplete}{\todo{Finish proof}}

\usepackage[backend=biber]{biblatex}
\addbibresource{bibliography.bib}

\usepackage{imakeidx}
\makeindex[intoc=true]

%\usepackage{glossaries-extra}
%\makeglossaries
%\input{chapters/chapter-2/chapter-2-glossary.tex}

\usepackage{amsmath}
\usepackage{amsfonts}
\usepackage{amssymb}
\usepackage{amsthm}
\usepackage{mathrsfs}

\theoremstyle{definition}
\newtheorem{theorem}{Theorem}[section]
\newtheorem{lemma}[theorem]{Lemma}
\newtheorem*{corollary}{Corollary}
\newtheorem{definition}[theorem]{Definition}
\newtheorem{pproposition}[theorem]{Proposition}
\newtheorem{exercise}{Exercise}[section]
\newtheorem*{proposition}{Proposition}
\newtheorem*{solution}{Solution}
\newtheorem*{remark}{Remark}

\usepackage{hyperref}

\title{Solutions Notebook for\\Calculus -- Early Transcendentals\\by James Stewart}
\author{Jose Fernando Lopez Fernandez}
\date{22 March, 2020 -- \today}

\begin{document}
	\frontmatter
	\maketitle
	\tableofcontents
	\mainmatter
%	\chapter{Functions and Models}
\section{Four Ways to Represent a Function}

\subsection{Exercises}
\begin{exercise}
	If $f\left(x\right)=x+\sqrt{2-x}$ and $g\left(u\right)=u+\sqrt{2-u}$, is it true that $f=g$?
\end{exercise}
\begin{solution}
	Yes. The $x$ and $u$ variables are free in $f$ and $g$, respectively, and thus both the image and pre-image of $f$ and $g$ are exactly the same.
\end{solution}

\begin{exercise}
	If $f\left(x\right)=\frac{x^2-x}{x-1}$ and $g\left(x\right)=x$, is it true that $f=g$?
\end{exercise}
\begin{solution}
	No. The two functions are not equal to one another because in order to be equal to one another, $f\left(x\right)$ must be equal to $g\left(x\right)$ for all values of $x$. In this particular case, $f$ has a discontinuity at $x-1$, and while the discontinuity is removable, $f$ itself is defined specifically with the discontinuity included, and thus cannot be equal to $g$.
\end{solution}

\begin{exercise}
	The graph of a function $f$ is given.
	\renewcommand{\labelenumi}{(\alph{enumi})}
	\begin{enumerate}
		\item State the value of $f\left(1\right)$.
		\item Estimate the value of $f\left(-1\right)$.
		\item For what values of $x$ is $f\left(x\right)=1$?
		\item Estimate the value of $x$ such that $f\left(x\right)=0$.
		\item State the domain and range of $f$.
		\item On what interval is $f$ increasing?
	\end{enumerate}
\end{exercise}
\begin{solution}
	{\color{white}{content...}}
	\renewcommand{\labelenumi}{(\alph{enumi})}
	\begin{enumerate}
		\item $f\left(1\right)=3$
		\item $f\left(-1\right)\approx 0.2$
		\item $\left\lbrace 0, 3 \right\rbrace$
		\item $f^{-1}\left(-1\right)\approx 0.7$
		\item Domain: $\left[ -2, 4 \right]$, Range: $\left[ -1, 3 \right]$
		\item $x \in \left[ -2, 1 \right)$
	\end{enumerate}
\end{solution}


\setcounter{section}{5}
\section{Inverse Functions and Logarithms}
\subsection{Exercises}
\begin{exercise}
	\index{one-to-one function}
	\index{function!one-to-one}
	{\color{white}{content...}}
	\renewcommand{\labelenumi}{(\alph{enumi})}
	\begin{enumerate}
		\item What is a one-to-one function?
		\item How can you tell from the graph of a function whether it is one-to-one?
	\end{enumerate}
\end{exercise}
\begin{solution}
	{\color{white}{content...}}
	\renewcommand{\labelenumi}{(\alph{enumi})}
	\begin{enumerate}
		\item A one-to-one function $f$ is a function that takes a single value $f\left(x\right)$ for each value of $x$. In other words, $f\left(x_1\right) = f\left(x_2\right) \leftrightarrow x_1 = x_2$. More formally, a one-to-one function is an injective morphism (i.e... a morphism whose kernel is trivial).
		\item The horizontal line test
	\end{enumerate}
\end{solution}


\section{Review}
\setcounter{subsection}{2}
\subsection{Exercises}
\setcounter{exercise}{22}
\input{chapters/chapter-1/section-7/exercises/exercise-23.tex}


\section{Principles of Problem Solving}
\setcounter{exercise}{11}
\begin{exercise}
	{\color{white}{content...}}
	\renewcommand{\labelenumi}{(\alph{enumi})}
	\begin{enumerate}
		\item Show that the function $f\left(x\right)=\ln\left(x+\sqrt{x^2+1}\right)$ is an odd function.
		\item Find the inverse function of $f$.
	\end{enumerate}
\end{exercise}
\begin{solution}
	{\color{white}{content...}}
	\renewcommand{\labelenumi}{(\alph{enumi})}
	\begin{enumerate}
		\item The derivative of the function is positive, and thus $f$ is strictly increasing over its entire domain, allowing us to conclude that $f$ is odd.
		\item The inverse function of $f$ is $\sinh x$.
	\end{enumerate}
\end{solution}



%	\chapter{Limits and Derivatives}
\setcounter{section}{5}
\section{Limits at Infinity; Horizontal Asymptotes}
\setcounter{theorem}{7}
\begin{definition}
	Let $f$ be a function defined on some interval $\left( -\infty, a \right)$. Then
	\begin{equation*}
	\lim\limits_{x \to -\infty} f\left(x\right) = L
	\end{equation*}
	means that for every $\epsilon > 0$ there is a corresponding number $N$ such that if $x < N$ then $\left| f\left(x\right) - L \right| < \epsilon$.
\end{definition}

\subsection{Exercises}
\setcounter{exercise}{70}
\begin{exercise}
	Use Definition~8 to prove that $\lim\limits_{x \to \infty} \frac{1}{x} = 0$.
\end{exercise}


\section{Derivatives and Rates of Change}
\begin{definition}[Tangent Line]
	The \textbf{tangent line} to the curve $y = f\left(x\right)$ at the point $P\left(a, f\left(a\right)\right)$ is the line through $P$ with slope
	\begin{equation}
	\label{equation-tangent-line}
	m = \lim\limits_{x \to a} \frac{f\left(x\right) - f\left(a\right)}{x - a}
	\end{equation}
\end{definition}

\subsection{Exercises}


%	\input{chapters/chapter-3.tex}
%	\input{chapters/chapter-4.tex}
%	\chapter{Integrals}
\setcounter{section}{3}
\section{Indefinite Integrals and the Net Change Theorem}
\begin{theorem}
	\label{net-change-theorem}
	The integral of a rate of change is the net change:
	\begin{equation}
	\int_{a}^{b} F^\prime\left( x \right) dx = F\left(b\right)-F\left(a\right)
	\end{equation}
\end{theorem}

\setcounter{problem}{20}
\begin{exercise}
	Evaluate the integral.
	\begin{equation*}
	\int_{-2}^{3} \left( x^2 - 3 \right) dx
	\end{equation*}
\end{exercise}

\setcounter{problem}{63}
\begin{exercise}
	Water flows from the bottom of a storage tank at a rate of $r\left(t\right)=200-4t$ liters per minute, where $0\leq~t\leq~50$. Find the amount of water that flows from the tank during the first $10$ minutes.
\end{exercise}
\begin{solution}
	The net change in the volume of water in the tank during the first ten minutes is equal to the integral of the rate at which the water was flowing out of the tank, from $t=0$ to $t=10$.
	\begin{align*}
	\Delta V&=\int_{0}^{10} r\left(t\right)dt\\
	&=\int_{0}^{10} 200-4t dt\\
	&=200\int_{0}^{10}dt -4\int_{0}^{10} t\: dt\\
	&=\left[200t\big|_0^{10}\right] -4\left[\frac{1}{2}t^2\big|_{0}^{10}\right] \\
	&=\left[200\left(10\right) - 200\left(0\right)\right]-4\left[\frac{1}{2}\left(10\right)^2 - \frac{1}{2}\left(0\right)^2\right]\\
	&=2000-4\left(50\right)\\
	&=2000-200\\
	&=1800
	\end{align*}
	Therefore, $1800$ liters of water flowed out of the tank in the first ten minutes.
\end{solution}



%	\chapter{Applications of Integration}
\setcounter{section}{3}
\section{Work}
\begin{exercise}
	A $360$-lb gorilla climbs a tree to a height of $20$ ft. Find the work done if the gorilla reaches that height in
	\renewcommand{\labelenumi}{(\alph{enumi})}
	\begin{enumerate}
		\item $10$ seconds
		\item $5$ seconds
	\end{enumerate}
\end{exercise}
\begin{solution}
	In both cases, the answer is the same.
	\renewcommand{\labelenumi}{(\alph{enumi})}
	\begin{enumerate}
		\item $7200$ foot-pounds
		\item $7200$ foot-pounds
	\end{enumerate}
	Force depends solely on the mass of the object and the distance it is being moved. Power, on the other hand, does depend on the amount of time it takes to do a certain amount of work.
\end{solution}

\begin{exercise}
	How much work is done when a hoist lifts a $200$-kg rock to a height of $3$ m?
\end{exercise}
\begin{solution}
	\begin{align*}
	\text{Work} &= \text{Force} \times \text{Distance}\\
	W &= \left( 200\text{kg} \cdot 9.8\frac{\text{m}}{\text{s}^2} \right) \times 3\text{m}\\
	&= 600 \frac{\text{kg}\cdot\text{m}\cdot\text{m}}{\text{s}^2}\\
	&= 600 \text{Nm}
	\end{align*}
\end{solution}



%	\input{chapters/chapter-7.tex}
%	\input{chapters/chapter-8.tex}
%	\input{chapters/chapter-9.tex}
%	\input{chapters/chapter-10.tex}
%	\appendix
%	\input{chapters/appendix-2.tex}
%	\input{chapters/appendix-3.tex}
%	\backmatter
%	\printbibliography[heading=bibintoc]
%	\printglossaries
%	\printindex
\end{document}
